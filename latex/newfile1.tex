%% LyX 2.1.4 created this file.  For more info, see http://www.lyx.org/.
%% Do not edit unless you really know what you are doing.
\documentclass[english]{article}
\usepackage[T1]{fontenc}
\usepackage[latin9]{inputenc}
\usepackage{amsmath}
\usepackage{amssymb}
\usepackage{babel}
\begin{document}

\title{Tightly-Coupled VI-EKF}


\author{James Jackson}

\maketitle

\section{$\boxplus$ and $\boxminus$ operators}
\begin{itemize}
\item \cite{key-1}Introduces a new syntax for working with manifold representation
of Lie Groups as if they were vectors.
\item Example: Quaternion Dynamics:
\end{itemize}
\begin{eqnarray}
\boldsymbol{q}_{t+1} & = & \boldsymbol{q}_{t}\boxplus\boldsymbol{\theta}\\
\boldsymbol{\theta} & = & \boldsymbol{q}_{1}\boxminus\boldsymbol{q}_{2}
\end{eqnarray}


It is important to note that the dimensionalities of $\boldsymbol{\theta}$
and $\boldsymbol{q}$ are different. The actual $\boxplus$ and $\boxminus$
operators are defined for robocentric quaternions representation as
follows:

\begin{align}
\boxplus & :SO\left(3\right)\times\mathbb{R}^{3}\rightarrow SO\left(3\right),\\
 & {\bf q},\boldsymbol{\theta}\mapsto{\bf q}\otimes\exp\left(\boldsymbol{\theta}\right)\\
\boxminus & :SO\left(3\right)\times SO\left(3\right)\rightarrow\mathbb{R}^{3},\\
 & {\bf q},{\bf p}\mapsto\log\left({\bf p}\otimes{\bf q}^{-1}\right),
\end{align}


This is a sort of weird syntax, but it allows us to work with these
parameterizations as if they were vectors. These operators become
the equivalent of vector addition and subtraction, and therefore allow
proper defintions of derivatives and integrals across these operators.
A properly defined $\boxplus$ manifold must obey the following identies

\begin{eqnarray}
 & x\boxplus0 & =x\\
\forall y\in S:\quad & x\boxplus\left(y\boxminus x\right) & =y\\
\forall\delta\in V:\quad & (x\boxplus\delta)\boxminus x & =\delta\\
\forall\delta_{1}\delta_{1}\in\mathbb{R}^{n}:\quad & \lVert(x\boxplus\delta_{1})\boxminus(x\boxplus\delta_{2})\rVert & \leq\lVert\delta_{1}-\delta_{2}\rVert
\end{eqnarray}


These operators must also form a diffeomorphism from $V$ to $S$,
so that derivatives of $\delta$ correspond to limits of $x\boxplus\delta$.
For example, the derivative of a quaternion, as defined using the
$\boxplus$ and $\boxminus$ operators are
\begin{eqnarray}
\dfrac{\partial}{\partial x}\boldsymbol{q}(x) & : & =\lim_{\epsilon\rightarrow0}\dfrac{\boldsymbol{q}(x+\epsilon)\boxminus\boldsymbol{q}(x)}{\epsilon}\\
\dfrac{\partial}{\partial\boldsymbol{q}}x(\boldsymbol{q}) & : & =\lim_{\epsilon\rightarrow0}\left[\begin{array}{c}
\dfrac{x\left(\boldsymbol{q}\boxplus(\boldsymbol{e}_{1}\epsilon)\right)-x\left(\boldsymbol{q}\right)}{\epsilon}\\
\dfrac{x\left(\boldsymbol{q}\boxplus(\boldsymbol{e}_{2}\epsilon)\right)-x\left(\boldsymbol{q}\right)}{\epsilon}\\
\dfrac{x\left(\boldsymbol{q}\boxplus(\boldsymbol{e}_{3}\epsilon)\right)-x\left(\boldsymbol{q}\right)}{\epsilon}
\end{array}\right]^{\top}
\end{eqnarray}


We can now use this operator to define dynamics and Jacobians across
our non-linear manifold representations.
\begin{thebibliography}{1}
\bibitem{key-1}Hertzberg, Christoph, et al. \textquotedbl{}Integrating
generic sensor fusion algorithms with sound state representations
through encapsulation of manifolds.\textquotedbl{} Information Fusion
14.1 (2013): 57-77.

\bibitem{key-2}Wheeler and Koch. \textquotedbl{}Derivation of the
Relative Multiplicative Extended Kalman Filter\textquotedbl{}

\bibitem{key-3}Beard, Randal W., and Timothy W. McLain. \textquotedbl{}Small
unmanned aircraft: Theory and practice\textquotedbl{}. Princeton university
press, 2012.

\bibitem{key-4}Bloesch, Michael Andre. State Estimation for Legged
Robots\textendash Kinematics, Inertial Sensing, and Computer Vision.
Diss. 2017.\end{thebibliography}

\end{document}
